\chapter{Conclusion and Project evaluation}
\label{chapterlabel7}

%waqas
This chapter completes the project by revising the project goals, objectives and personal aims in order to assess whether they were met. An evaluation of the project examines the processes and procedures that were implemented to develop and deliver the project. Finally, future work and and concluding remarks are presented.

\section{Project Goals}

The introductory goals of the project have been defined quite broadly on purpose since there was substantial uncertainty  in the initial stage of the project about how quickly the development of the application would proceed. The author is generally satisfied with the results, although some components were more successful than the rest. The first project goal was to develop and deliver a web application with a clear and usable interface that will enable the user to register and upload all the relevant information of an operation, including the generated from the operation media files. Sensor Fusion Web Application has been successfully deployed and tested at \url{https://sensorfusion.azurewebsites.net/} and all the aforementioned functionalities have been included. The second goal was to build the back-end part of the web application, that will make it able to extract all the meta-data from the selected media files, store the media files in a private file storage and store all the information of the operation in a relational database. That was arguably the hardest part of the application, but the application now fully supports the extraction,storing and retrieval of all the metadata related to the media files. The second goal was to develop a combination of front-end and back-end components that will allow the user to search for an operation stored in the relational database, apply certain filters to reduce the displayed results in order to find an operation of their choice. This goal has also been achieved, as the home page of the application demonstrates the listing and searching capabilities of the web application. The third and final goal of the application was to design a distinct web page in the application where the user is able to view all the aggregated information that is related to a specific operation. The web application includes a separate ``details'' page where the user can all the relevant information of an operation, including the media files' URL and metadata information. Overall, the author's opinion is that the development of the application went wery well. During the development process of the server side, there was never an occasion where where a part of the application had to be restructured or reconstructed in order to make another part of the application work properly, which by itself was a strong sign of separation of concerns.

\section{Fulfilment of Personal Aims}

The personal aims of this project were mostly focused on learning more and becoming proficient in building and developing dynamic web applications. In regards to the server side language, the programming language of choice was C\# and ASP.NET Core framework, as described throughout the current report. During the development phase and after the first weeks of learning C\# and ASP.NET, it was only able to build static pages and some basic database manipulation. Nonetheless, after a intense period, the author was able to realise the true potential of the framework and from then on the development stage progressed at a rapid pace. Overall, the learning experience was very intensive but at the same time very rewarding, and thus this particular stage can be regarded as one of the most significant personal achievements of the whole project. Feeling confident to develop robust web application in a new programming language like C\# and being able to build an application from scratch in the ASP.NET framework was certainly a fulfilment of one of my personal aims. By undertaking this particular project, the front-end programming skills of the author have been significantly improved. After the completion of the project, the author gained a significant amount of knowledge on creating attractive user interfaces and is comfortable to a certain level to understand the proper use of JavaScript and HTML5. Furthermore, the use of jQuery and Ajax methods helped with the development of a firm understanding of how JavaScript can be used in order to create a rich and interactive website. Finally, the last achieved personal aim was to build onn previous knowledge of MySQL database. By creating the web application, it provided the opportunity to further improve into a very advanced level the knowledge that had been gained through the GC04 and COMPGC06 database modules. The application helped develop and refine the skills needed to create arguable the most elaborate, robust and stable database schema explained in detail in Chapter \ref{chapterlabel3}.


\textbf{textHere}\\

\textbf{textHere}\\

\textbf{textHere}\\

\textbf{textHere}\\

\textbf{textHere}\\




\section{Critical Evaluation}


\section{Future Work}


\section{Conclusion}



