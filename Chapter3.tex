\chapter{Requirements and Analysis}
\label{chapterlabel3}


This Chapter establishes the requirements and analysis for this work. The developments of the requirements is a crucial part of the analysis for almost all of the software development projects.

This phase is extremely useful, as for it's flexibility - if things that are unnecessarily complicated are discovered, or are less important to the client, they can be disregarded  or their importance could be downgraded so that they take less time. This is significantly useful for the implementation phase. This is also the easiest time to communicate with the client, while the subject matter is strictly textual, so any major issues or changes to the specifications can be established here.


\subsection{Problem Statement}
\label{sub:problem_st}

Design a web based application, that allows surgeons to register an operation, upload and view all the extracted data coming from the operating theatre's sensors.


\section{Software Requirement Listing and Prioritization}
\label{sec:softreqlistandprior} 

This stage facilitates a comprehensive understanding of the client's needs and to appreciate the potential complexity of the system. Subsection \ref{sub:functional_req} details the functional requirements of our system.  Functional requirements show the behaviours of the system - what the client wants the system to do.  Subsection \ref{sub:non_functional_req} goes through the non-functional requirements, which affect the system's performance. 
The requirements were gathered through discussion with the client and also by continually iterating over what the application would be used for, so that additional focus should be given on the user needs.


 As mentioned, the requirements have been categorised into functional and non-functional. They have also been prioritised according to the MoSCoW prioritisation technique\cite{cleggbarker1994}, in order to guide the progress of the project and to ensure that a base-level application that achieved the goals of the project. The MoSCoW priorities refer to the order of design and development in order.




\subsection{Functional Requirements}
\label{sub:functional_req}

\newcounter{magicrownumbers}
\newcommand\rownumber{\stepcounter{magicrownumbers}\arabic{magicrownumbers}}
{
\footnotesize
%\centering
\begin{longtable}{|p{0.5cm}|p{13cm}p{1.3cm}|}

\rowcolor[HTML]{000000}
{\color[HTML]{FFFFFF} \textbf{ID}} & {\color[HTML]{FFFFFF} \textbf{Functional Requirements}}  & {\color[HTML]{FFFFFF} \textbf{Priority}} \\ \hline \endhead
\multicolumn{3}{|c|}{\textbf{Uploading a new Operation}} \\ \hline
\rownumber & The platform shall support a User Interface that will allow the user to register a new operation &Must  \\ \hline
\rownumber & The platform shall be able to take as input from the user, the hospital that the operation took place &Must  \\ \hline
\rownumber & The platform shall be able to take as input from the user, the operating room that the operation took place &Must  \\ \hline
\rownumber & The platform shall be able to take as input from the user, all the staff that participated in the operation &Must  \\ \hline
\rownumber & The platform shall be able to take as input from the user, the type of the operation (Neurosurgery, Hand surgery, Paediatric surgery etc.) &Must  \\ \hline
\rownumber & The platform shall be able to take as input from the user, the specific patient that has undergone the surgery including the patients unique identification number &Must  \\ \hline
\rownumber & The platform shall be able to take as input from the user, all the video files that were produced during the operation &Must  \\ \hline
\rownumber & The platform shall be able to take as input from the user, all the audio files that were produced during the operation &Must  \\ \hline
\rownumber & The platform shall be able to take as input from the user, the file extracted from the patient's monitoring system &Should  \\ \hline
\rownumber & The platform shall be able to record, store and distinguish data from different operating theatres &Must  \\ \hline
\rownumber & The platform shall be able to store all the input data from the user to a relational database &Must  \\ \hline
\rownumber & The platform shall store all the data coming from an operation in a way that all relevant data of the operation could be extracted &Must  \\ \hline
\rownumber & The platform shall be able to process the video, audio and patients monitoring files uploaded from the user &Must  \\ \hline
\rownumber & The platform shall be able to extract all the meta-data from the input files (encoded date, size, duration, file type, file name, full file path) &Must  \\ \hline
\rownumber & The platform shall be able to store the meta-data extracted from the input files, to the relational database &Must  \\ \hline
\rownumber & The platform shall be able to store the input files to an Azure Blob Storage Account &Must  \\ \hline
\rownumber & The platform shall be able to link the files stored in the Azure Blob Storage with the relative operation identification number stored in the relational database &Must  \\ \hline


\multicolumn{3}{|c|}{\textbf{Searching for an Operation}} \\ \hline
\rownumber & The platform shall support searching capabilities; the user of the platform must be able to search an operation/procedure using relevant criteria  &Must  \\ \hline
\rownumber & The platform shall support filtering capabilities where the user can apply specific filters for an operation (hospital name, operating room number, from/to date, doctors name, patient's name, type) &Must  \\ \hline

\multicolumn{3}{|c|}{\textbf{Details of an Operation}} \\ \hline
\rownumber & The platform shall be able to display a specific page where the user can see all the relative information and details of a specific operation &Must  \\ \hline
\rownumber & The platform shall be able to retrieve all the relevant data from the operation (video data, microphone data, patient's monitoring system data etc.) &Must  \\ \hline
\rownumber & The platform shall be able to convert the data coming from the different sensors to easily handled format ( e.g. convert a variety of video input to .avi) &Could  \\ \hline
\rownumber & The platform shall support the capability of switching to a specific moment in time and view all the recorded data at that moment &Could  \\ \hline
\rownumber & The platform shall be able to present to the user all the data recorded from the sensors at the same time. For example it could be able to view the panoramic camera, the light camera, the heart rate, the temperature of the room through the common factor of time. &Could  \\ \hline
\rownumber & The platform shall provide the media files' URL to the user in order for the user to have access to them and re-play them &Must  \\ \hline

\caption[Functional Requirements]{Functional Requirements} % needs to go inside longtable environment
\label{functionalreq}
\end{longtable}
}

\subsection{Non Functional Requirements}
\label{sub:non_functional_req}

{\footnotesize
%\newcommand\rownumber{\stepcounter{magicrownumbers}\arabic{magicrownumbers}}
%\centering
\begin{longtable}{|p{0.5cm}|p{13cm}p{1.3cm}|}

\rowcolor[HTML]{000000}
{\color[HTML]{FFFFFF} \textbf{ID}} &{\color[HTML]{FFFFFF} \textbf{Non-Functional Requirements}}  & {\color[HTML]{FFFFFF} \textbf{Priority}} \\ \hline \endhead
\rownumber & The platform shall use web browser as its user interface & Must \\ \hline
\rownumber & The platform shall support new sensor adding or should require minimal work for new, unknown sensor adding & Should \\ \hline
\rownumber & The platform shall store all the input data in a way that they could be extracted for machine learning purposes (machine learn-able data) & Must \\ \hline
\rownumber & The platform shall be independent of the format of the input data & Should  \\ \hline
\rownumber & The platform shall work on a wide range of operating conditions (screen size, internet connection, performance) & Must \\ \hline
\rownumber & The platform shall be a C\# web application with an MySQL Database & Should \\ \hline
\rownumber & The System shall present search results within 5 seconds & Should \\ \hline

\caption[Non-Functional Requirements]{Non-Functional Requirements} % needs to go inside longtable environment
\label{nonfunctionalreq}
\end{longtable}
}


\section{Domain Modelling}
Following the analysis and establishment of the functional and non-functional requirements in Sections 2.2
and 2.3, the domain model now presents the conceptual model of the system’s problem domain. To be more
precise, the classes in this model were initially retrieved from the requirements list, where the identified entities
are highlighted in bold. A simple domain model draft was created from these entities and duplicates were
identified and eliminated. Furthermore, items which were deemed unnecessary from this iteration were also
removed to refine the list of entities. This initial model was then revised and additional domain objects were
added. Similar to previous analyses, the domain model is therefore the result of an iterative process.

Following the refinement procedure, using the identified entities of the requirements list, the team revised
the domain model to establish any further domain objects that weren’t in the requirements. The resulting model,
following the various iterations, is shown in Figure 2.1. It summarises the responsibilities of the system on a general level and was then used to construct the use cases in Chapter 3. Attention was paid to the fact that
external systems should be modelled as external actors [7], as can be seen from the payment system actor in
the diagram.


\section{Use Case Analysis}
\blindtext

\section{Use Case Diagram}

\section{Object Oriented Analysis}

% This just dumps some pseudolatin in so you can see some text in place.


\blindtext
\blindtext

